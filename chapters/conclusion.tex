\chapter{Conclusion}

The thesis dealt with the problem of an automatic language grammar import into the MPS editor.
As stated in the first chapter (\ref{chap:main_goals}), it is not expected, that this thesis will solve the problem of the grammar import completely.
It is expected, that more follow-up work will be based on this thesis and will further improve the import process.
Main goals of the thesis~(\ref{chap:main_goals}) were presumably achieved, even though it is quite hard to set acceptance boundaries, since it is arguable, what level of perfection was expected.
\\

The author explored given environment~(chapters~\ref{chap:jetbrains_mps} and~\ref{chap:related_projects} and made some technological decisions, such as choosing the grammar notation~\ref{chap:source_grammar_notation}).
These decisions proved to be good ones, since they enabled the author to achieve further goals quite efficiently.
By \textit{further goals} we mean implementing an MPS plugin, that imports given ANTLRv4 grammar as an MPS language.
The resulting MPS language is usable and, after some customization, can be used as a solid foundation of a full MPS language.
Some simple languages, such as JSON or JavaScript, yield very good results, comparable to existing manual ports, but sparing the user from a lot of tedious and time consuming work.
\\

It is also hard to measure the success, because before the work on this thesis had started, it has been quite unclear, what can actually be achieved.
For instance, it was presumed (both by JetBrains and the supervisor), that the biggest part of the research would be devoted to the problem of projectional editor generation.
It was expected, we will spend a lot of time figuring out, how to derive code layout from the grammar.
However, a major part of the effort was put into the structure aspect, which turned out to hold some interesting problems on its own.
Some of these problems have been overcome and some remain unresolved.
The most important problems, that we were unable to solve, concern the structure of the language and are described in chapter \ref{chap:problems_with_grammars}.
The source of these problems comes from the fact, that we use the grammar for a little bit different purpose, than it was intended to.
Parsing existing source code and creating rules for writing correct code are two different points of view.
These problems also present potential obstacles for future follow-up work, so we believe, it is very important, that we have identified and described them.
