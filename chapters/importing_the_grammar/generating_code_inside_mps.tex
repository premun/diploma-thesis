\section{Generating code inside MPS}
\label{chap:generating_code_inside_mps}

In order to create some of the aspects of the imported language such as the editor or the textgen aspect, we will need to generate code inside MPS. For all coding inside MPS the BaseLang language is used, which is the official JetBrains port of Java for MPS. MPS is also self-hosting because it is written partially in Java but mostly in BaseLang itself.
\\

Because we are inside MPS and BaseLang is an MPS language, this code is not a text source code, but again, an AST built out of concepts belonging to the BaseLang language (and more subsidiary MPS languages that bring the MPS API and bring some additional syntax). This means that generating code inside MPS consists of generating some AST out of BaseLang nodes. It is a very interesting problem since it is a little bit more challenging than just generating plain text code. For example to generate a simple statement, where we instantiate an object into an object's property:

\begin{center}
	\texttt{foo.bar = new node<Element{\_}1>();}
\end{center}

\noindent
we need to do a whole lot of things. We need to find the declaration of the foo variable, look up its property bar and tell the MPS we would like to build an assignment statement with left side containing some sort of a property-access-expression and a new statement with some specific type on the right. The full AST tree, that needs to be built in order to create a statement like this, is shown in figure \ref{fig:new_ast}.

\begin{figure}[hb]
	\centering
	\hspace*{-18mm}
	\includegraphics[width=200mm]{../img/new_statement_ast.png}
	\caption{BaseLang "New statement" abstract syntax tree}
	\label{fig:new_ast}
\end{figure}

\subsection{Quotation}
The example above showed, how many nodes are need in order to represent quite basic simple statement. Luckily for us, the developers of MPS have thought about this and implemented a special notation that helps building expressions. This notation is called quotation and it enables us to insert statements into a special quoted block, similarly like an eval function can parse string into a real code. Furthermore, inside this quotation block we can insert an anti-quotation from which we can reference variables from our scope outside of the original quotation. This is what the above statements looks like when quotation (light blue) and anti-quotation blocks (yellow) are used:

\begin{center}
	\texttt{\colorbox{cyan!30}{<}\lexerrule{\%(} fooRef \lexerrule{)\%}.\lexerrule{\%(} barProp \lexerrule{)\%} = new node <\lexerrule{\textasciicircum(} type \lexerrule{)\textasciicircum}>();>\colorbox{cyan!30}{>}}
\end{center}

\subsection{Generating dynamic code}

Quotation comes in handy when we are generating code statically. Meaning we know, in the time of implementation, what code we want to generate. But what if our code depends on input data? There are several parts where the generated BaseLang code depends solely on our grammar's structure. In those cases we cannot simply use quotation but we are left with generating statement's tree node by node as in the first example. Quotation can make some of these parts a little bit shorter, but doesn't solve the problem entirely.