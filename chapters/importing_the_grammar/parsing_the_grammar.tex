\section{Parsing the grammar}

First stage of the import process is parsing the grammar file. The parser must be able to read any grammar file and extract the structure information. As stated in the chapter defending the grammar notation selection (chapter  \ref{chap:source_grammar_notation}), there exists an ANTLRv4 grammar for the ANTLRv4 notation\footnote{https://github.com/antlr/grammars-v4/tree/master/antlr4}. This means that we can use the ANTLR library\footnote{http://antlr.org/} to generate an ANTLR parser of the ANTLRv4 notation. 

\subsection{Grammar representation}

The tree that comes out of the automatically generated parser is a little bit complicated and full of information not relevant to us. It also contains all structures including ANTLR's special characters (pipes, semicolons, comments...), because ANTLR doesn't understand the grammar, just parses it. That is why we decided to translate this complex structure into our own simplified tree made out of our own objects. Of course, that we could somehow skip this step but since we are working on a difficult problem by itself, we aimed for simplicity and solution clarity. These objects are tailored exactly to our use case and hold as little information as possible. Furthermore, we process the grammar in two iterations. In the first we build a tree holding names of other referenced rules in the form of strings and in the second we walk this tree and resolve all these references to real pointers to other rule objects. We end up with a neat representation that is easy to parse in next steps of the import process.

\subsection{Flattening lexer rules}
One of the questions we had to deal with was lexer rule representation. Not minding all the syntactic sugar ANTLR is offering for their definition, in the end, lexer rules are basically just regular expressions used for parsing input into tokens. Later in MPS, we would like to assign these expressions to some constraint entities. However, after the initial parsing we end up with some kind of a tree structure that is representing the lexer rule. 

\pagebreak

Let's look at the lexer rule \lexerrule{Name}, that we have in our SimpleXML language, together with its sub rules:

\begin{antlr}
	\lexerrule{Name}         :   \lexerrule{NameStartChar} \lexerrule{NameChar}* ;
	
	\textbf{fragment}
	\lexerrule{DIGIT}        :   \regex{[0-9]} ;
	
	\textbf{fragment}
	\lexerrule{NameChar}     :   \lexerrule{NameStartChar}
	|   \literal{-} | \literal{_} | \literal{.}
	|   \lexerrule{DIGIT}
	;
	
	\textbf{fragment}
	\lexerrule{NameStartChar}:   \regex{[:a-zA-Z]} ;
\end{antlr}

We can notice that we could flatten this complex rule into a single regular expression. We can glue elements of each alternative together and then join these alternatives with a classic regex OR ($|$), which is exactly its function in the grammar notation.
\\

We have put these thoughts into a form of the following recursive algorithm:    

\begin{antlr}
	Flatten(\textit{R}):
	1) Define \textit{T} as a list of empty string
	2) For each alternative \textit{A} of the rule \textit{R}:
	3)     Define \textit{R} = \ap\ap
	4)     For each element \textit{E} of \textit{A}:
	5)         If \textit{E} is not yet flattened:
	6)             Flatten(\textit{E})
	7)         \textit{R}.append(\textit{E})
	8)         \textit{R}.append(\textit{E}.operator)
	9)     \textit{T}.add(\textit{R})
	10) Build a string S out of elements t\textsubscript{1}, t\textsubscript{2}, ... t\textsubscript{n} of \textit{T}:
	11)     (t\textsubscript{1})|(t\textsubscript{2})|...|(t\textsubscript{n})
	12) Return \textit{S}
\end{antlr}

The output of \texttt{Flatten(Name)} then would return a regular expression in the form of:

\begin{center}
	\texttt{([:a-zA-Z])(([:a-zA-Z])|(-)|({\_})|(\.)|([0-9]))*}
\end{center}

We have made some further adjustments which for example remove unnecessary braces and added some other transformations, because the regular expression notation of ANTLR is not the same as in Java, which is the one we are targeting. These are some minor changes such as character exclusion etc. Special characters also had to be escaped, because MPS holds regular expressions in the form of a string and not all escape sequences are allowed.

\subsection{Subrule inline blocks}

A new feature was added to the fourth version of ANTLR notation that makes parsing the structure a little bit more complicated. This feature allows to inline some rules directly inside an alternative. Following excerpt from the official XML's ANTLRv4 grammar definition shows this extended syntax:

\begin{antlr}
	\parserrule{content} :   \lexerrule{TEXT}? ((\parserrule{element}? | \lexerrule{CDATA} | \lexerrule{COMMENT}) \lexerrule{TEXT}?)* ;
\end{antlr}

There is a nested block contained in the second part of the rule that has another block nested inside. Even though the content rule contains only one alternative, it is possible to build its contents from more variations. There can be any number of levels of these nested blocks and each block can be annotated with its own quantification operator. 
\\

We solved this complication by recursive traversal of the parser tree. We decided to generate a new parser rule for each such block and then reference this rule from its original place. The full expanded content rule then looks like this:

\begin{antlr}
	\parserrule{content} :   \lexerrule{TEXT}? \parserrule{block_1}* ;
	
	\parserrule{block_1} :   \parserrule{block_2} \lexerrule{TEXT}? ;
	
	\parserrule{block_2} :   \parserrule{element}? 
	|   \lexerrule{CDATA} 
	|   \lexerrule{COMMENT}
	;
\end{antlr}