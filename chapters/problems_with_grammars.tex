\chapter{Problems with grammars}
\label{chap:problems_with_grammars}

As we could see in previous chapters, it is very easy to write the grammar in a way that will cause problems once it is imported into MPS. These problems might occur during creation of any aspect and they are very hard to be mitigated. As an example we can look at the \parserrule{content} rule of the original XML grammar:

\begin{antlr}
	\parserrule{content} :   \lexerrule{TEXT}? ((\parserrule{element}? | \lexerrule{CDATA} | \lexerrule{COMMENT}) \lexerrule{TEXT}?)* ;
\end{antlr}

As we have shown in the TextGen chapter \ref{chap:textgen} [TODO], the author of the grammar decided to handle whitespace not by skipping it using ANTLR actions but rather by wrapping other content with a rule that contains these characters. Then, using a clever block rule, we make sure that any following content might or might not be separated with whitespace/text too.
\\

This, however, will cause mess when imported inside MPS. We have no other choice than break up the inner block and create three concepts, holding two placeholders respectively for:

\begin{itemize}
	\item An \parserrule{element} and \lexerrule{TEXT}
	\item A \lexerrule{CDATA} section and \lexerrule{TEXT}
	\item A \lexerrule{COMMENT} and \lexerrule{TEXT}
\end{itemize}

After that, we create the \concept{Content} concept containing one \lexerrule{TEXT} and a reference to any of these three newly created concepts.
\\

When we will start using this language, it will be full of (most likely) empty \lexerrule{TEXT} placeholders, sometimes holding texts, sometimes spaces, but mostly distracting us and polluting the editor. There is however no way how we could detect or prevent this when reading the grammar, since we do not really understand the content.
\\

We cannot detect any smart shortcut since all blocks are different concepts for us entirely. We could of course try to parse the block differently in the first place and somehow for example create an interface for the most inner block and group the three block concepts into one, but because blocks can be nested inside of each other freely and there can be any setup within multiple levels additionally with various quantitative operators, we will always find a very simple case that will break our algorithm.
\\

Additionally, there might be other problems such as layering, which we tried to prevent, but it can still occur in some form. There are also some cases where we do not have a big problem per se, but altering the grammar might yield better MPS languages. We will show one more example where we don't want to improve the structure, but again, the usability of the resulting MPS language. Let's look at the original XML attribute and its value:

\begin{antlr}
	\parserrule{attribute}   :   \lexerrule{Name} \literal{=} \lexerrule{STRING} ;

	\lexerrule{STRING}      :   \literal{"} \regex{~["]*} \literal{"}
	            |   \literal{\textbackslash'} \regex{~[']*} \literal{\textbackslash'}
	            ;
\end{antlr}

The original XML grammar has quotes as a part of the value. For the resulting MPS language, it would mean that there would be a placeholder for the attribute value, that would expect us to input the leading and trailing quote together with the value too each time. It would also be displayed in red unless we enter both quotes since the regex will not match. The user won't be able to tell why, though. We could, however, adjust the grammar easily in following manner:

\begin{antlr}
	\parserrule{attribute}   :   \lexerrule{Name} \literal{="} \lexerrule{TEXT1} \literal{"}
	            |   \lexerrule{Name} \literal{=\textbackslash'} \lexerrule{TEXT2} \literal{\textbackslash'}
	            ;

	\lexerrule{TEXT1}       :   \regex{~["]*} ;
	\lexerrule{TEXT2}       :   \regex{~[']*} ;
\end{antlr}

We turned quotes into literals and they will only appear in the projectional editor part. We won't have to write them each time. This however has its drawbacks because user will have to choose, which attribute version he wants to use, decreasing the fast with which we could type code. We could also decide to drop one of the two options, but the we will no longer have a full port of the language. Or we could improve the imported language to use one as default and add a special action to switch between these two. We could also allow any content inside as MPS would handle values containing both quotes and apostrophes, but again, we would differentiate from the original. It is hard to decide which solution to this problem is the best. Users might have different intentions, requirements or goals with their MPS language and all approaches lead to some results.

\section{Adjusting grammars}
We conclude that changing the grammar itself automatically is even harder problem. Writing the grammar itself is a challenging task by itself even for a human and with all its complexities such as modes and actions, it would be impossible to understand the grammar enough to mangle with its structure without breaking it.
\\

We could however see, that a simple adjustment, like those we did in our SimpleXML language, can eradicate all of these problems and leave us with a nicely usable MPS language. We also think that the end user of our plugin will be quite educated in this field as they will already be trying to create an MPS language. It is also expected that this user will continue on improving the language after the initial import, because the language will just not be as usable without the human touch. From these reasons we conclude that  the end user might also be capable to perform similar grammar adjustments themselves.

\section{Problems with altered grammars}
As stated before, when creating our own version of simplified XML, we have started off with the XML grammar and did some adjustments just as described previously. After we ended up with a grammar just good enough for our plugin, we noticed that even though the imported language behaves well enough, ANTLR parser created from this grammar no longer parses XML successfully. For this, we can blame the ANTLR parser and some dangerous lexer rules. The greedy way, in which ANTLR matches input on defined tokens and prioritizes their selection, makes some rules (such as \lexerrule{TEXT}) very dangerous, because they swallow a lot more characters than they should. It is very easy to slip and perform an adjustment that will at first look seem valid inside MPS, but will break the parser.
\\

The reason, why we care about not making the grammar dysfunctional like this, is that if we wanted to proceed with some more complex operations, we would need to automatically generate parsers and use them somehow to parse existing code in that language. This could be for example utilized within the projectional editor, when detecting layout, or when trying to load existing source code and opening it inside MPS. The user might not even notice that the grammar was damaged or it might have been corrupt from the very beginning. And sometimes we might even wish to break it just for the sake of a better outcome of the import itself (attribute example above). But, advancing further without a valid grammar might make things very complicated.
